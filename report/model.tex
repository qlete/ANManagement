\section{Model}
\label{sec:model}

In this section, we present the model we used to solve the problem of ANM in DSO grid.
In particular, we will show how this problem can be cast as a MDP and how RL can be included when the distribution of the uncertainty is unknown and we will highlight the modelling assumptions used.

The electrical network can be mathematically modeled as a graph whose edges represent the electrical lines and whose nodes are locations in the network where power can be consumed or injected.
In a typical distribution network, the nodes correspond to households or small factories that behave as prosumers.
In this work, we will make the assumption that the graph is a tree as it is the case of most distribution networks.
Let $\mathcal{N}$ be the set of nodes and $\mathcal{E}$ be the set of edges.
Let $\mathcal{G}$ be the graph representing the network, i.e. $\mathcal{G} = (\mathcal{N}, \mathcal{E})$.

We represent every node by an integer in $\{0, \dots, n\}$ where $n = |\mathcal{N}|-1$ is the cardinality of set $\mathcal{N}$.
The root of the tree is the node that connects the distribution network to the transmission network and is the node $0$ in our model and we denote by $\mathcal{N}_+ = \mathcal{N} \backslash \{0\}$.
Each node $i \in \mathcal{N}_+$ has a unique ancester denoted by $A_i$ and each node $i \in \mathcal{N}$ has a set of children denoted by $\mathcal{C}_i$.
We choose the orientation of the lines from being from $i$ to $A_i$ so that we can unambiguously represent each line by its origin node.
We thus have $\mathcal{E} = \{1, \dots, n\}$.
We will define the following nodal and branch quantities :

For each node $i \in \mathcal{N}$, we define :
\begin{itemize}
  \item $v_i$ as the square of the norm of the complex voltage at that node,
  \item $s_i = p_i + \mathbf{i} q_i$ as complex net power injection (the net injection being the power consumed minus the power produced)
  \item $d_i$ is the active net power injection demanded at node $i$, i.e. the power that the agent at that node wants to reinject in the network.
  It is thus different than the power that will actually be seen by the network (that is, $p_{i}$) as some curtailment mechanisms can take place.
  This will be further discussed later.
  It is positive when the power is injected and negative if it is retreived.
\end{itemize}

For each line $i \in \mathcal{E}$, we define :
\begin{itemize}
  \item $z_i = r_i + \mathbf{i}x_i$ as the complex impedence,
  \item $l_i$ as the square of the norm of the complex current,
  \item $S_i = P_i + \mathbf{i}Q_i$ as the sending-end complex power, where $P_i$ denote the active power and $Q_i$ the reactive power.
\end{itemize}

A schematic representation of a line together with the notations we use is given below (Figure \ref{fig:line}).

\begin{figure}[H]
\centering
\begin{tikzpicture}[scale=8]
    \draw [->] (0,0.7) -- (0,0.53);
    \draw [->] (1,0.7) -- (1,0.53);
    \draw [thick] (0,0.5) -- (1, 0.5);
    \draw [<-] (0.35, 0.45) -- (0.65, 0.45);
    \node [left] at (0,0.7) {$s_{A_i}$};
    \node [right] at (1,0.7) {$s_i$};
    \node [below, left] at (0,0.5) {$v_{A_i}$};
    \node [below, right] at (1,0.5) {$v_i$};
    \node [above] at (0.5, 0.5) {$z_i$};
    \node [below] at (0.4, 0.45) {$l_i$};
    \node [below] at (0.6, 0.45) {$S_i$};
\end{tikzpicture}
\caption{Schematic representation of a line.}
\label{fig:line}
\end{figure}

We consider the problem in a multi-time-step framework.
This means that the variables are defined at each time step.
We consider a horizon of $T$ time steps.
The set of time steps is denoted by $\mathcal{T}$ and each one is represented by a positive integer, i.e. $\mathcal{T} = \{1, \dots, T\}$.
