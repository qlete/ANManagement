\section{Conclusion and further work} % (fold)
\label{sec:further_work}
% The work will be based on scientific papers or books and should contain both a theoretical contribution (description of the method) as well as an empirical validation. The balance between the two components depends on your wish and expectation: the orientation of the work can be more theoretical or more practical. If, for instance, you decide to face a complex application, you can limit yourself to a description of the basic algorithms you used (MDP, RL or BP). Conversely, if you decide to study a theoretical extension, the practical application can just be the “Snakes and Ladders” game investigated in the first project.


We designed a new layout for seeing nodes as uncertainty and edges as actions
regarding which houses to consider.

Considering \emph{further work}, we would design a solution in which
the link between the actions and the states is more profound
in a sense that the nodes should include more information about what
has been done in the last action and from what state it comes.
Trying to figure out the actual distribution of probabilities,
if such exists, could also be an interesting path to follow.
Finally it might also be interesting to see what happens if a lot of houses
are considered in the model as well as more a wider range of choice in the rate of curtailment.
% section further_work (end)

