\section{Introduction}
\label{sec:intro}
With the advent of Renewable Energy Sources (RES) for electricity production at the distribution scale, the role of the Distribution System Operator (DSO) is expected to change dramatically in the next decade. 
Historically, the flow of power in the distribution network was simple and well defined : acquired from the Transmission grid, the power was injected in the network at the substation node and was sent to the residential consumer located at the root of the network. 
Nowadays, power can be produced by the residential customers themselves who reinject their overproduction in the network.
The flow of power became bi-directional, which implies new challenges in the operations of the distribution network (DN). 
Moreover, the energy produced at the distribution scale is also intermittent and has a high uncertainty. 
This uncertainty should be taken into account and it is necessary for the DSO to develop new tools for an optimal operations of the network.
In its report about the energy transition \cite{utility}, the MIT Energy Initiative insist on the fact that DSOs will have to abandon their old \textquote{fit-and-forget} approach, which consist in decoupling the planning and operations of the network. 
With their own words : 
\begin{displayquote}
Distribution companies will have to become true \textquote{system operators}.
\end{displayquote}

In this context, one of the main challenges for the DSO is to maintain a high reliability. 
This imply that integrating RES in the system should have no influence on the voltage and current control. 
Voltage should remain below an acceptable deviation from its nominal value to ensure the security of electric devices and current should remain below the line capacity. 
Active Network Management (ANM) is a precious tool to combine optimal operations with high reliability. 
In this work, we will use reinforcement learning to include the uncertainty in existing ANM tools. 

The report is organized as follows :